% Options.
%
% Language
%  en  -  english (default)
%  de  -  german
%
% Modes
%  draft     - drafting mode, esp for todos
%  fulldraft - drafting mode, also for listings/graphics, faster
%  final     - final mode
%
%
% Fonts
%  lmodern   - Latin Modern, like TeX default
%  palatino  - Heavier, slightly more readable (default)
%  garamond  - Lighter than palatino, heavier than latin modern, more readable.
%
% Other
%  print - do not color links
%  zebralistings - alternating row colors for listings
%
\documentclass[draft,master]{swathesis}
\usepackage[backend=biber]{biblatex}
\addbibresource{thesis.bib}

\usepackage{titlepage}
\TitlePageStyle[
subject=master,degree={Master of Science},
]{hpi-swa}

\supervisors{%
Prof.\,Dr.\,Jack Hasaname\and %
Dr.\,John Doe%
}

% ABGABEDATUM
% \setdate{2012}{04}{05}
% \date{\datedate}

\author{Vorname Nachname}
\location{Berlin}
\extratitle{\raggedleft Nachname, \\ Titel Kurz\par}
\title{Der Lange Titel der Arbeit die zu Schreiben Ist}
\subtitle{Ein Untertitel ist Optional}
\othertitle{The Long Title of the Work to Be Done}
\othersubtitle{This is Optionally Optional}

\begin{document}
\frontmatter
\maketitle
The english abstract.

%%% Local Variables:
%%% mode: latex
%%% End:

\begingroup
\let\raggedsection\centering

\chapter*{Acknowledgments}
\label{cha:acknowledgments}
\endgroup
\begin{quotation}
  \noindent I owe everything to my cat.
\end{quotation}
%%% Local Variables:
%%% mode: latex
%%% End:

\tableofcontents
\listoffigures
\listoftables
\lstlistoflistings
\listofacronyms %
\mainmatter
% HAUPTTEIL
% \input{introduction}
% \input{context}
% \input{solution}
% \input{implementation}
% \input{evaluation}
% \input{relatedwork}
% \input{conclusion}
%

% beispiel.

\chapter{Introduction}
\label{cha:introduction}

\blindtext
\cite{Kay:2011:PCC:800193.1971922}

\secmissing{Motivation}

\todolist{Outline}{
\item intro
\item problem
\item solution
\item implementation
\item eval
\item related
\item summary
}

\todosec{I am missing a section here}

Here goes the foo.\todo{cite here}
I will talk about what an \ac{api} is in this section.
Normally, there are different \acp{api}. Not to forget,
anything can be an \ac{api}. And full: \acf{api}.

\begin{code}[lst:example]{An example code snippet}{float,numbers=left,language=Smalltalk}
exampleWithNumber: x

"A method that illustrates every part of Smalltalk method syntax
except primitives. It has unary, binary, and keyword messages,
declares arguments and temporaries, accesses a global variable
(but not and instance variable), uses literals (array, character,
symbol, string, integer, float), uses the pseudo variables
true false, nil, self, and super, and has sequence, assignment,
return and cascade. It has both zero argument and one argument blocks."

    |y|
    true & false not & (nil isNil) ifFalse: [self halt].
    y := self size + super size.
    #( #a "a" 1 1.0)
        do: [:each | Transcript show: (each class name);
                                 show: ' '].
     ^ x < y
\end{code}

\begin{table}
  \centering
  \tableinsidecommand
  \begin{threeparttable}
    \caption{Differences between things projected and things achieved}
    \label{tab:things}
    \begin{tabular}{>{}p{.4\linewidth}@{}c} \toprule
      Part           & done        \\ \midrule
      Title          & \y          \\
      Abstract       & \n          \\
      Intro          & \y          \\
      \multicolumn{2}{c}{Rest is not entirely true} \\ \midrule
      Context        & \y          \\
      Problem        & \n\tnote{a} \\
      Solution       & \y          \\
      Implementation & \y          \\
      Evaluation     & \n          \\
      Related Work   & \n          \\
      Conclusion     & \y          \\ \bottomrule
    \end{tabular}
    \begin{tablenotes}
      \item [a] Just few things missing
    \end{tablenotes}
  \end{threeparttable}
\end{table}

We have nice things: some code in \autoref{lst:example} and some information
in \autoref{tab:things}.



\[\nabla \cdot \mathbf{E} = \frac{\rho}{\varepsilon_0}\]
\[\nabla \cdot \mathbf{B} = 0\]
\[\nabla \times \mathbf{E} = -\frac {\partial \mathbf{B}}{\partial t}\]
\[\nabla \times \mathbf{B} = \mu_0 \mathbf{J} + \mu_0\varepsilon_0  \frac{\partial \mathbf{E}}{\partial t}\]


\missingfigure{Make a overview sketch of the whole system}


\blinddocument

\printbibliography
\clearpage
\appendix
%% ggf:
% \part{Appendix}
% \label{part:appendix}

\chapter{First Unimportant stuff.}
\label{cha:first-unimp-stuff}

\blindtext
%%% Local Variables: 
%%% mode: latex
%%% End: 

\backmatter
\markboth{}\relax

% BAMA-O (2009) §24.8
%  Am Schluss der Arbeit hat die/der Kandidat/in  zu versichern, dass 
%  sie/er sie selbstständig verfasst sowie keine anderen Quellen und 
%  Hilfsmittel als die angegebenen benutzt hat.
% BAMA-O (2013) §30.6
%  Am Schluss der Arbeit hat die Kandidatin bzw. der Kandidat zu versichern,
%  dass sie bzw. er die Arbeit selbstständig verfasst und keine anderen
%  Quellen und Hilfsmittel als die angegebenen benutzt hat.
\defaultstatement
\end{document}
%%% Local Variables:
%%% mode: latex
%%% TeX-master: t
%%% End:
