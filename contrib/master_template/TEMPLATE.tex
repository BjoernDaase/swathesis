%
%   When TeX complains about microtype:
%
%  Installation MikTeX: see contrib/microtype.tar.xz
%     http://docs.miktex.org/manual/localadditions.html
%  (also for TeXLive <2011)
%
%
%  TeXLive 2011: 
%     tlmgr --repository http://tlcontrib.metatex.org/2011/ install microtype
%  TeXLive 2012: 
%     tlmgr --repository http://tlcontrib.metatex.org/2012/ install microtype


% Options.
%
% Language
%  en  -  english (default)
%  de  -  german
%
% Modes
%  draft - drafting mode, esp for todos
%  final - final mode
%
%
% Fonts
%  lmodern - Latin Modern, like TeX default (default)
%  palatino -  Heavier, slightly more readable
%                      Palatino Linotype, Palatino or TeX Gyre Pagella  (xetex)
%                      TeX Gyre Pagella (luatex)
%                      Palladio+Pazo (pdflatex)
%  garamond - Lighter than palatino, heavier than latin modern, 
%                         more readable.
%                      needs EB Garamond:
%                       http://www.georgduffner.at/ebgaramond/
%                          download:
%            https://github.com/downloads/georgd/EB-Garamond/EBGaramond-0.013.zip
%                         unzip and install (xetex) or set
%                         \garamondpath (luatex)
%
% Other
%  print - do not color links
%
\documentclass[draft,master]{swathesis}

\addbibresource{thesis.bib}

\usepackage{titlepage}
\TitlePageStyle[
subject=master,degree={Master of Science},
]{hpi-swa}

\supervisors{%
Prof.\,Dr.\,Jack Hasaname\and %
Dr.\,John Doe%
}

% ABGABEDATUM
% \setdate{2012}{04}{05}
% \date{\datedate}

\author{Vorname Nachname}
\location{Berlin}
\extratitle{\raggedleft Nachname, \\ Titel Kurz\par}
\title{Der Lange Titel der Arbeit die zu Schreiben Ist}
\subtitle{Ein Untertitel ist Optional}

\begin{document}
\frontmatter
\maketitle
The english abstract.

%%% Local Variables:
%%% mode: latex
%%% End:

\begingroup
\let\raggedsection\centering

\chapter*{Acknowledgments}
\label{cha:acknowledgments}
\endgroup
\begin{quotation}
  \noindent I owe everything to my cat.
\end{quotation}
%%% Local Variables:
%%% mode: latex
%%% End:

\tableofcontents
\listoffigures
\listoftables
\lstlistoflistings
\listofacronyms %
\mainmatter
% HAUPTTEIL
% \input{introduction}
% \input{context}
% \input{solution}
% \input{implementation}
% \input{evaluation}
% \input{relatedwork}
% \input{conclusion}
%

% beispiel.

\chapter{Introduction}
\label{cha:introduction}

\blindtext
\cite{Kay:2011:PCC:800193.1971922}

\secmissing{Motivation}

\todolist{Outline}{
\item intro
\item problem
\item solution
\item implementation
\item eval
\item related
\item summary
}

\todosec{I am missing a section here}

Here goes the foo.\todo{cite here}
I will talk about what an \ac{api} is in this section.
Normally, there are different \acp{api}. Not to forget,
anything can be an \ac{api}. And full: \acf{api}.

\blinddocument

\printbibliography
\clearpage
\appendix
%% ggf:
% \part{Appendix}
% \label{part:appendix}

\chapter{First Unimportant stuff.}
\label{cha:first-unimp-stuff}

\blindtext
%%% Local Variables: 
%%% mode: latex
%%% End: 

\backmatter
\markboth{}\relax
\defaultstatement
\end{document}
%%% Local Variables:
%%% mode: latex
%%% TeX-master: t
%%% End:
