%%
%% content.tex
%%
\authorrunning{Author}
\titlerunning{The importance of work}
\title{The importance of why and how to do work}
\subtitle{An imaginary paper to showcase a document}
\extratitle{\raggedleft Author\\ Importance of Work}
\author{Anton Author}
\institute{%
  Hasso-Plattner-Institut, Potsdam\\
  \email{\{firstname.lastname\}@student.hpi.uni-potsdam.de}
}
\supervisors{%
  Prof.\,Dr.\,William Withaname \and%
 Dr.\,John Doe}
\maketitle

\begin{abstract}
  Hello, here is some text without a meaning. This text should show what a
  printed text will look like at this place. If you read this text, you will
  get no information. Really? Is there no information? Is there a difference
  between this text and some nonsense like “Huardest gefburn”? Kjift – not at
  all! A blind text like this gives you information about the selected font,
  how the letters are written and an impression of the look. This text should
  contain all letters of the alphabet and it should be written in of the
  original language. There is no need for special content, but the length of
  words should match the language.
\end{abstract}

\section{Introduction}
\label{sec:introduction}

Hello, here is some text without a meaning. This text should show what a
printed text will look like at this place. If you read this text, you will get
no information. Really? Is there no information? Is there a difference between
this text and some nonsense like “Huardest gefburn”? Kjift – not at all! A
blind text like this gives you information about the selected font, how the
letters are written and an impression of the look. This text should contain all
letters of the alphabet and it should be written in of the original language.
There is no need for special content, but the length of words should match the
language.\todo{cite}

Hello, here is some text without a meaning. This text should show what a
printed text will look like at this place. If you read this text, you will get
no information. Really? Is there no information? Is there a difference between
this text and some nonsense like “Huardest gefburn”? Kjift – not at all! A
blind text like this gives you information about the selected font, how the
letters are written and an impression of the look. This text should contain all
letters of the alphabet and it should be written in of the original language.
There is no need for special content, but the length of words should match the
language.

\section{Context}
\label{sec:context}

Hello, here is some text without a meaning. This text should show what a
printed text will look like at this place. If you read this text, you will get
no information. Really? Is there no information? Is there a difference between
this text and some nonsense like “Huardest gefburn”? Kjift – not at all! A
blind text like this gives you information about the selected font, how the
letters are written and an impression of the look. This text should contain all
letters of the alphabet and it should be written in of the original language.
There is no need for special content, but the length of words should match the
language.

\todosec{Is that redundant?}

Hello, here is some text without a meaning. This text should show what a
printed text will look like at this place. If you read this text, you will get
no information. Really? Is there no information? Is there a difference between
this text and some nonsense like “Huardest gefburn”? Kjift – not at all! A
blind text like this gives you information about the selected font, how the
letters are written and an impression of the look. This text should contain all
letters of the alphabet and it should be written in of the original language.
There is no need for special content, but the length of words should match the
language.

\section{Problem}
\label{sec:problem}

\secmissing{The Main Problem}

\section{Solution}
\label{sec:solution}

\todolist{anton}{
\item Put A onto B
\item Put B into C
\item Pull D from C
}

\section{Implementation}
\label{sec:implementation}

\missingfigure{Make a overview sketch of the whole system}

\section{Evaluation}
\label{sec:evaluation}

\lstset{language=Smalltalk}
\iffalse
\begin{verbatim}\fi
\begin{code}[lst:example]{An example code snippet}{float,numbers=left}
exampleWithNumber: x

"A method that illustrates every part of Smalltalk method syntax
except primitives. It has unary, binary, and keyword messages,
declares arguments and temporaries, accesses a global variable
(but not and instance variable), uses literals (array, character,
symbol, string, integer, float), uses the pseudo variables
true false, nil, self, and super, and has sequence, assignment,
return and cascade. It has both zero argument and one argument blocks."

    |y|
    true & false not & (nil isNil) ifFalse: [self halt].
    y := self size + super size.
    #($a #a "a" 1 1.0)
        do: [:each | Transcript show: (each class name);
                                 show: ' '].
     ^ x < y
\end{code}
\iffalse
\end{verbatim}\fi



\section{Related Work}
\label{sec:related-work}

\section{Conclusion}
\label{sec:conclusion}

\ctable%
[table,
caption={Differences between things projected and things achieved}
label=tbl:things]%
{>{}p{.4\linewidth}@{}c}%
{\tnote[a]{Just few things missing}}%
{%
\FL Part           & done
\ML Title          & \y
\NN Abstract       & \n
\NN Intro          & \y
\NN \multicolumn{2}{c}{Rest is not entirely true}
\ML Context        & \y
\NN Problem        & \n\tmark[a]
\NN Solution       & \y
\NN Implementation & \y
\NN Evaluation     & \n
\NN Related Work   & \n
\NN Conclusion     & \y
\LL
}


\nocite{*}
\bibliography{references}

% BAMA-O §20.7 
% Ist die Arbeit in einer 
%Fremdsprache verfasst, muss sie als Anhang eine 
% kurze Zusammenfassung in deutscher Sprache enthalten.
\begin{zusammenfassung}
  Weit hinten, hinter den Wortbergen, fern der Länder Vokalien und Konsonantien
  leben die Blindtexte. Abgeschieden wohnen Sie in Buchstabhausen an der Küste
  des Semantik, eines großen Sprachozeans. Ein kleines Bächlein namens Duden
  fließt durch ihren Ort und versorgt sie mit den nötigen Regelialien. Es ist
  ein paradiesmatisches Land, in dem einem gebratene Satzteile in den Mund
  fliegen. Nicht einmal von der allmächtigen Interpunktion werden die
  Blindtexte beherrscht – ein geradezu unorthographisches Leben. Eines Tages
  aber beschloß eine kleine Zeile Blindtext, ihr Name war Lorem Ipsum, hinaus
  zu gehen in die weite Grammatik. Der große Oxmox riet ihr davon ab, da es
  dort wimmele von bösen Kommata, wilden Fragezeichen und hinterhältigen
  Semikoli, doch das Blindtextchen ließ sich nicht beirren. Es packte seine
  sieben Versalien, schob sich sein Initial in den Gürtel und machte sich auf
  den Weg. Als es die ersten Hügel des Kursivgebirges erklommen hatte, warf es
  einen letzten Blick zurück auf die Skyline seiner Heimatstadt Buchstabhausen,
  die Headline von Alphabetdorf und die Subline seiner eigenen Straße, der
  Zeilengasse. Wehmütig lief ihm eine rhetorische Frage über die Wange, dann
  setzte es seinen Weg fort. Unterwegs traf es eine Copy
\end{zusammenfassung}

%%% Local Variables: 
%%% mode: latex
%%% End: 
